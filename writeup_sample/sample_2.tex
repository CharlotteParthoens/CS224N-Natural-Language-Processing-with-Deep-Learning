\documentclass[]{article}

%%%%%%%%%%%%%%%%%%%
% Packages/Macros %
%%%%%%%%%%%%%%%%%%%
\usepackage{amssymb,latexsym,amsmath}     % Standard packages
\usepackage{graphicx}

%%%%%%%%%%%
% Margins %
%%%%%%%%%%%
\addtolength{\textwidth}{1.0in}
\addtolength{\textheight}{1.00in}
\addtolength{\evensidemargin}{-0.75in}
\addtolength{\oddsidemargin}{-0.75in}
\addtolength{\topmargin}{-.50in}


%%%%%%%%%%%%%%%%%%%%%%%%%%%%%%
% Theorem/Proof Environments %
%%%%%%%%%%%%%%%%%%%%%%%%%%%%%%
\newtheorem{theorem}{Theorem}
\newenvironment{proof}{\noindent{\bf Proof:}}{$\hfill \Box$ \vspace{10pt}}  


%%%%%%%%%%%%
% Document %
%%%%%%%%%%%%
\begin{document}

\title{Sample \LaTeX ~File}
\author{David P. Little}
\maketitle

\begin{abstract}
This document represents the output from the file ``sample.tex" once compiled using your favorite \LaTeX compiler.  This file should serve as a good example of the basic structure of a ``.tex" file as well as many of the most basic commands needed for typesetting documents involving mathematical symbols and expressions.  For more of a description on how each command works, please consult the links found on our course webpage.
\end{abstract}



\section{Lists}
%%%%%%%%%%%%%%%
\begin{enumerate}
\item {\bf First Point (Bold Face)}
\item {\em Second Point (Italic)}
\item {\Large Third Point (Large Font)}
    \begin{enumerate}
        \item {\small First Subpoint (Small Font)} 
        \item {\tiny Second Subpoint (Tiny Font)} 
        \item {\Huge Third Subpoint (Huge Font)} 
    \end{enumerate}
\item[$\bullet$] {\sf Bullet Point (Sans Serif)}
\item[$\circ$] {\sc Circle Point (Small Caps)} 
\end{enumerate}
\section{Nonsense}

\centerline{\sc \large A Simple Sample \LaTeX\ File}
\vspace{.5pc}
\centerline{\sc Stupid Stuff I Wish Someone Had Told Me Four Years Ago}
\centerline{\it (Read the .tex file along with this or it won't 
            make much sense)}
\vspace{2pc}

The first thing to realize about \LaTeX\ is that it is not ``WYSIWYG''. 
In other words, it isn't a word processor; what you type into your 
.tex file is not what you'll see in your .dvi file.  For example, 
\LaTeX\ will      completely     ignore               extra
spaces    within                             a line of your .tex file.
Pressing return
in 
the 
middle 
of
a
line
will not register in your .dvi file. However, a double carriage-return
is read as a paragraph break.

Like this.  But any carriage-returns after the first two will be 
completely ignored; in other words, you 


can't 

add






more 




space 


between 




lines, no matter how many times you press return in your .tex file.

In order to add vertical space you have to use ``vspace''; for example, 
you could add an inch of space by typing \verb|\vspace{1in}|, like this:
\vspace{1in}

To get three lines of space you would type \verb|\vspace{3pc}|
(``pc'' stands for ``pica'', a font-relative size), like this:
\vspace{3pc}

Notice that \LaTeX\ commands are always preceeded by a backslash.  
Some commands, like \verb|\vspace|, take arguments (here, a length) in
curly brackets.  

The second important thing to notice about \LaTeX\ is that you type 
in various ``environments''...so far we've just been typing regular 
text (except for a few inescapable usages of \verb|\verb| and the
centered, smallcaps, large title).  There are basically two ways that 
you can enter and/or exit an environment;
\vspace{1pc}

\centerline{this is the first way...}

\begin{center}
this is the second way.
\end{center}

\noindent Actually there is one more way, used above; for example, 
{\sc this way}.  The way that you get in and out of environment varies
depending on which kind of environment you want; for example, you use 
\verb|\underline| ``outside'', but \verb|\it| ``inside''; 
notice \underline{this} versus {\it this}.

The real power of \LaTeX\ (for us) is in the math environment. You 
push and pop out of the math environment by typing \verb|$|. For 
example, $2x^3 - 1 = 5$ is typed between dollar signs as
\verb|$2x^3 - 1 = 5$|. Perhaps a more interesting example is
$\lim_{N \to \infty} \sum_{k=1}^N f(t_k) \Delta t$.

You can get a fancier, display-style math 
environment by enclosing your equation with double dollar signs.  
This will center your equation, and display sub- and super-scripts in 
a more readable fashion:

$$\lim_{N \to \infty} \sum_{k=1}^N f(t_k) \Delta t.$$

If you don't want your equation to be centered, but you want the nice 
indicies and all that, you can use \verb|\displaystyle| and get your 
formula ``in-line''; using our example this is 
$\displaystyle \lim_{N \to \infty} \sum_{k=1}^N f(t_k) \Delta t.$  Of 
course this can screw up your line spacing a little bit.

There are many more things to know about \LaTeX\ and we can't 
possibly talk about them all here.
You can use \LaTeX\ to get tables, commutative diagrams, figures, 
aligned equations, cross-references, labels, matrices, and all manner 
of strange things into your documents.  You can control margins, 
spacing, alignment, {\it et cetera} to higher degrees of accuracy than 
the human eye can percieve.  You can waste entire days typesetting 
documents to be ``just so''.  In short, \LaTeX\ rules.

The best way to learn \LaTeX\ is by example. Get yourself a bunch
of .tex files, see what kind of output they produce, and figure out how
to modify them to do what you want.  There are many template and 
sample files on the department \LaTeX\ page and in real life in the 
big binder that should be in the computer lab somewhere.  Good luck!


\section{student}

\leftline{Pat Q.~Student}
\leftline{AME 20231}
\leftline{18 January 2013}

\medskip
This is a sample file in the text formatter \LaTeX.
I require you to use it for the following reasons:

\begin{itemize}

\item{It produces the best output of text, figures,
      and equations of any
      program I've seen.}

\item{It is machine-independent. It runs on Linux, Macintosh (see {\tt TeXShop}), and Windows (see {\tt MiKTeX}) machines.
     You can e-mail {\tt ASCII} text versions of most relevant files.}

\item{It is the tool of choice for many research
     scientists and engineers.
     Many journals accept 
     \LaTeX~ submissions, and many books
     are written in \LaTeX.}

\end{itemize}
\medskip
Some basic instructions are given next.
Put your text in here.  You can be a little sloppy    about
spacing.  It adjusts the text to look good.
{\small You can make the text smaller.}
{\tiny You can make the text tiny.}

Skip a line for a new paragraph.   
You can use italics ({\em e.g.} {\em  Thermodynamics is everywhere}) or {\bf bold}.
Greek letters are a snap: $\Psi$, $\psi$,
$\Phi$, $\phi$.  Equations within text are easy---
A well known Maxwell thermodynamic relation is
$\left.{\partial T \over \partial p}\right|_{s} = 
\left.{\partial v \over \partial s}\right|_{p}$.
You can also set aside equations like so:
\begin{eqnarray}
du &=& Tds -p dv, \qquad \mbox{first law}\\
ds &\ge& {dq \over T}.\qquad  \qquad \mbox{second law} \label{ee}
\end{eqnarray}
Eq.~(\ref{ee}) is the second law.
References\footnote{Lamport, L., 1986, {\em \LaTeX: User's Guide \& Reference Manual},
    Addison-Wesley: Reading, Massachusetts.}
are available. 
If you have an postscript file, say {\tt sample.figure.eps}, in the same local directory,
you can insert the file as a figure.  Figure \ref{sample}, below, plots an isotherm for air modeled as an ideal gas. 


\medskip
\leftline{\em Running \LaTeX}
\medskip

You can create a \LaTeX~ file with any text editor ({\tt vi}, {\tt emacs}, {\tt gedit}, 
etc.). 
To get a document, you need to run the \LaTeX~ application
on the text file.  The text file must have the suffix ``{\tt .tex}''
On a Linux cluster machine, this is done via the command

\medskip
{\tt latex file.tex}

\medskip
\noindent
This generates three files: {\tt file.dvi}, {\tt file.aux},
and {\tt file.log}.  The most important is {\tt file.dvi}. 

\medskip
\noindent
The finished product can be previewed in the following way.
Execute the commands:

\medskip

{\tt dvipdf file.dvi}\hspace{1.9in}{\em Linux System}

\medskip
\noindent
This command generates {\tt file.pdf}.  
Alternatively, you can use {\tt TeXShop} on a Macintosh or {\tt MiKTeX} on a Windows-based machine.
The {\tt .tex} file must have a closing statement as
below.



\section{Equations}
%%%%%%%%%%%%%%%%%%%

\subsection{Binomial Theorem}
\begin{theorem}[Binomial Theorem]
For any nonnegative integer $n$, we have
$$(1+x)^n = \sum_{i=0}^n {n \choose i} x^i$$
\end{theorem}

\subsection{Taylor Series}
The Taylor series expansion for the function $e^x$ is given by
\begin{equation}
e^x = 1 + x + \frac{x^2}{2} + \frac{x^3}{6} + \cdots = \sum_{n\geq 0} \frac{x^n}{n!}
\end{equation}


\subsection{Sets}

\begin{theorem}
For any sets $A$, $B$ and $C$, we have
$$ (A\cup B)-(C-A) = A \cup (B-C)$$
\end{theorem}

\begin{proof}
\begin{eqnarray*}
(A\cup B)-(C-A) &=& (A\cup B) \cap (C-A)^c\\
&=& (A\cup B) \cap (C \cap A^c)^c \\
&=& (A\cup B) \cap (C^c \cup A) \\
&=& A \cup (B\cap C^c) \\
&=& A \cup (B-C)
\end{eqnarray*}
\end{proof}


\section{Tables}
%%%%%%%%%%%%%%%%
\begin{center}
\begin{tabular}{l||c|r}
left justified & center & right justified \\ \hline
1 & 3.14159 & 5 \\
2.4678 & 3 &  1234 \\ \hline \hline
3.4678 & 6.14159 & 1239
\end{tabular}
\end{center}


\section{A Picture}
%%%%%%%%%%%%%%%%%%%
\begin{center}
\begin{picture}(100,100)(0,0)
\setlength{\unitlength}{1pt}
\put(20,70){\circle{30}}  \put(20,70){\circle*{10}}   % left eye
\put(80,70){\circle{30}}  \put(80,70){\circle*{10}}   % right eye
\put(40,40){\line(1,2){10}} \put(60,40){\line(-1,2){10}} \put(40,40){\line(1,0){20}} % nose
\put(50,20){\oval(80,10)[b]} % mouth
\multiput(0,90)(4,0){10}{\line(1,3){4}}  % left eyebrow
\multiput(100,90)(-4,0){10}{\line(-1,3){4}}  % right eyebrow
\end{picture}
\end{center}


\end{document}